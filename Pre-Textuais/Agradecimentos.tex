% Agradecimentos
% ---
\begin{agradecimentos}

É praxe fazer os agradecimentos e é muita falta de educação não os fazer. A gratidão “é uma virtude fundamental nas relações humanas”, pois, de acordo com Ortega y Gasset, a pessoa ingrata “se esquece de que a maior parte daquilo que tem não é obra sua, mas de outros, que se esforçaram para criá-las e obtê-las. Assim, ao esquecer-se disso, ele despreza radicalmente a verdadeira condição do que tem” (TESCAROLO, 2004).

É praxe e educado iniciar os agradecimentos para a pessoa que orientou o trabalho.

Escrever os nomes e sobrenomes completos e corretos de todos os que efetivamente colaboraram no trabalho. Lembrar que as pessoas têm família, ou seja, sobrenomes. Por exemplo: agradecemos ao Sr. Fulano da firma de aditivos que gentilmente nos cedeu as amostras para o trabalho; perceber que isso é uma construção grosseira e, portanto, deve ser evitada. Se houver pessoa jurídica, o nome da empresa e o cargo da pessoa devem constar nos agradecimentos. Cuidar para que pessoas do corpo técnico que colaboraram sejam devidamente citadas com seus cargos e nomes completos. Se o uso e costume exigir que para a correta identificação de uma pessoa seja utilizado outro nome, ele deverá ser colocado entre parênteses, após o nome completo, como no exemplo: ...agradecemos à laboratorista Sra. Maria de Lurdes Sobrenome (D. Malu) etc.

Normalmente, as pessoas às quais desejamos expressar maior gratidão são citadas ao final.

Evitar clichês. Clichês são frases batidas que denotam desleixo na elaboração do texto. Exemplo: Por fim, agradecemos a todos sem os quais esse trabalho não seria possível. Mesmo sendo verdadeiro, isso é um clichê e deve ser evitado, pois esvazia o texto.


\end{agradecimentos}