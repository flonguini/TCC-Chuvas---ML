% RESUMOS
% ---

% resumo em português
\setlength{\absparsep}{18pt} % ajusta o espaçamento dos parágrafos do resumo

\begin{resumo}
Manter as equações de chuvas intensas atualizadas para todas as regiões do Brasil é um grande desafio devido ao fato da grande quantidade de equações, além de passar por um processo complexo de análise estatística dos dados. Esse trabalho estuda a aplicação de redes neurais artificias no problema de criar uma equação que represente o mais fidedignamente possível as características pluviométricas de uma região a partir de dados históricos de chuvas. Para esse objetivo, foram analisados duas arquiteturas de redes recorrentes que possibilitassem criar um algoritmo universal com a capacidade de representar, e entender, as mudanças de qualquer região do mundo a partir de uma série histórica. Os resultados obtidos foram satisfatórios, obtendo uma acurácia que descreve melhor a região do que os métodos estatísticos utilizados atualmente. O modelo conseguiu chegar a coeficientes de determinação de até 0,9973 para uma rede com 10 neurônios em uma camada e um período de retorno de 10 anos, apresentando um coeficiente mínimo de 0,9562 para um período de retorno de 20 anos e 1 neurônio artificial.

\textbf{Palavras chave:} Rede Neural Artificial. Redes Neurais Recorrentes. Aprendizado de Máquina. Equações IDF.

\end{resumo}

% resumo em inglês
\begin{resumo}[Abstract]
    \foreignlanguage{english}
    {
        \textit{
        To keep the intense rain equations updated for all regions of Brazil is a great challenge due to the large number of equations, besides going through a complex process of statistical analysis of the data. This thesis studies the application of artificial neural networks in the problem of creating an equation that represents as accurately as possible the pluviometric characteristics of a region from historical rainfall data. For this purpose, two architectures of recurrent neural networks were analyzed that allowed to create a universal algorithm with the ability to represent and understand the changes of any region of the world from a historical series. The results were satisfactory obtaining an accuracy that better describes the region than the statistical methods currently used. The model was able to obtain coefficients of determination up to 0.9973 for a network with 10 neurons in one layer and a return period of 10 years, presenting a minimum coefficient of 0.9562 for a return period of 20 years and 1 artificial neuron.
        }
    }
    \vspace{\onelineskip}
    \noindent 
    \textbf{\\Keywords:} Artificial Neural Network. Recurrent Neural Networks. Machine Learning. IDF Equations.

\end{resumo}