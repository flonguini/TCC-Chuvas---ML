% RESUMOS
% ---

% resumo em português
\setlength{\absparsep}{18pt} % ajusta o espaçamento dos parágrafos do resumo
\begin{resumo}
 É o penúltimo texto a ser elaborado. É feito em um único parágrafo contendo de 150 até 500 palavras (ABNT NBR 6028, 2003) (Como fazer no Word: Guia Revisão, Grupo Revisão de Texto, Comando Contar palavras). O resumo deve ser do tipo informativo e conter informações sobre a introdução, fundamentos teóricos, material, métodos, resultados, discussão e conclusões. Se alguém quiser ler somente o resumo, deverá ser possível conhecer o trabalho em sua essência. Após o texto do resumo são colocadas até cinco palavras-chave conforme o modelo abaixo: palavra-chave, ponto; palavra-chave, ponto; até cinco. Serão as mesmas utilizadas como descritores do trabalho na ficha catalográfica (nome atual: dados internacionais de catalogação-na-publicação). Lembrar que esse será o texto mais lido de seu trabalho e, em muitos casos, o único texto a ser lido. Para fazer um bom resumo, seguir a orientação dada no Modelo de construção do resumo (GLASMAN-DEAL, 2010).
 
\textbf{Palavras chave:} Palavra chave1. Palavra chave2. Palavra chave3. Palavra chave4. Palavra chave5. (As palavras-chaves são as MESMAS que serão utilizadas na ficha catalográfica)

Modelo de construção do RESUMO:

Frases 1-2: O autor fornece informações factuais que servem de subsídio para a compreensão do trabalho.

Frase 2: O autor apresenta o objetivo geral, o objetivo específico do trabalho e o método, preferencialmente numa única frase. (Talvez sejam necessárias duas frases, pois isso depende de prática).

Frases 3-4: O autor resume a metodologia e fornece detalhes do que foi estudado.

Frases 5-7: O autor descreve os principais resultados do trabalho.

Frase 8: O autor apresenta as conclusões do trabalho 

\end{resumo}

% resumo em inglês
\begin{resumo}[Abstract]
    \foreignlanguage{english}
    {
        \textit{
        Este é o último texto a ser elaborado. O Abstract é o resumo vertido para a língua inglesa. O texto deve ser em itálico. É conveniente criar um estilo Abstract para que se possa tirar proveito do dicionário de inglês interno do Word. O estilo desejável para esse texto é o de Inglês universitário (o que não é fácil). Sugestão: utilizar o site: http://www.phrasebank.manchester.ac.uk/index.htm para construir estruturas de frases em inglês genuinamente universitário.
        }
    }
    \vspace{\onelineskip}
    \noindent 
    \textbf{\\Keywords:} Keyword1. Keyword2. Keyword3.(Traduções das palavras-chaves utilizadas no resumo e na ficha catalográfica)

\end{resumo}