\section{Determinação da equação IDF}

De acordo com (VILLELA e MATOS, 1975), deve-se analisar as relações entre intensidade-duração-frequência das chuvas, determinando para diferentes intervalos de duração, qual o tipo de equação e o número de parâmetros que melhor representam a região. A equação escolhida para esse projeto será a equação geral, devido a adoção por diversos autores.

Para obter as séries históricas de máximas precipitações, verificou-se, para a PCD escolhida, a máxima precipitação anual para cada intervalo de duração. A duração de 5 minutos, adotou-se, conforme (RAMOS, 2010), a intensidade mínima de 8 milímetros.

Após a determinação das máximas precipitações anuais, será encontrada a melhor distribuição empírica de probabilidade que se ajuda aos dados.
Para realizar a extrapolação dos dados será utilizada a distribuição estatística para valores máximos extremos de Gumbel.

\subsection{Teste de aderência}

Será realizado um teste estatístico para validar se a distribuição adotada realmente representa os dados empíricos. Como comentado no capítulo anterior, em algumas regiões do Brasil não são comuns o monitoramento histórico de chuvas, sendo admissível a utilização de períodos inferiores ao recomendado mediante a alguns testes de validação. Dentre esses, o teste de Kolmogorov-Smirnov é utilizado para testar e validar o ajuste de distribuições contínuas.

\subsection{Obtenção dos parâmetros da equação IDF}

Para calcular os coeficientes k, a, b e c da Equação 1, é necessário estimar o valor do parâmetro b, pois a partir de um grupo de valores, são adotados metodologias para estimar os demais parâmetros. Brater (1964) sugere que sejam arbitrados valores de b até que, os pontos plotados graficamente, tornem-se uma reta. Utilizando um gráfico bi logaritmo adota-se o eixo X sendo as durações e o eixo Y as intensidades calculadas pela distribuição de Gumbel.
Aplicando-se a função log10(x) na Equação 1, obtém-se a Equação 31.

Dessa forma, é possível ajustar, pelo método da regressão linear múltipla, os demais parâmetros da equação.