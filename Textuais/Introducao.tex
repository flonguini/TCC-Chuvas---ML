\textual

% ----------------------------------------------------------
% Introdução (exemplo de capítulo sem numeração, mas presente no Sumário)
% ----------------------------------------------------------
\chapter*[Introdução]{Introdução}
\addcontentsline{toc}{chapter}{Introdução}
% ----------------------------------------------------------

A precipitação pluvial é a variável meteorológica de destaque no ciclo hidrológico, pois é a responsável pela entrada de água nas bacias hidrográficas. Sua importância é facilmente compreensível quando se considera o papel da água na vida humana devido os efeitos catastróficos das grandes cheias e estiagens (PINTO, HOLTZ, et al., 1976).

O estudo de precipitações máximas é indispensável para o correto dimensionamento de obras hidráulicas como sarjetas, bocas de lobo, bueiros, pontes, piscinões, vertedouros e galerias. A chuva extrema máxima é aquela que apresenta grande precipitação, em um curto intervalo de tempo, a importância do seu estudo para quantificar, adequadamente, seus efeitos uma vez que a mesma intensidade pode causar distintas consequências em diferentes ambientes.  De acordo com (CRUCIANI, MACHADO e SENTELHAS, 2002), a caracterização das chuvas intensas é imprescindível para que seus efeitos sejam verificados adequadamente, além de assegurar a previsão da ocorrência de eventos hidrológicos extremos e suas consequências da forma mais real e precisa possível.

Para caracterização das chuvas intensas, utilizam-se modelos matemáticos que definem sua intensidade, duração e frequência — denominadas curvas IDF —, a estatística dos valores extremos, segundo Chow-Gumbel, apresenta-se como um dos métodos mais indicados para caracterizar a distribuição das chuvas máximas. (SILVA, 2016) Assim, procura-se analisar as relações intensidade-duração-frequência das chuvas observadas, determinando-se para os diferentes intervalos de duração da chuva, qual o tipo de equação e qual o número de parâmetros para essa equação que melhor caracterizam aquelas relações.  (VILLELA e MATOS, 1975)

A determinação das curvas IDF utilizando séries históricas de precipitações obtidas por dados de postos de coleta apresenta grande dificuldade em razão da baixa densidade de postos de coleta e do pequeno período de observações que estão disponíveis para consulta pública. 

Devido a novas tecnologias computacionais, a inteligência artificial tem facilitado a análise e tomada de decisões em diversas áreas. Essa ciência nasceu a partir do reconhecimento de padrões e na teoria de que computadores podem aprender sem serem programados para performar uma tarefa em específico. Pesquisadores na área de inteligência artificial gostariam de saber se computadores podem aprender a partir de dados.  O aspecto interativo do aprendizado de máquina é importante, pois os modelos são expostos a novos dados constantemente, e eles são capazes de se ajustar. Os algoritmos aprendem de processamentos anteriores para produzir informações confiantes, decisões repetidas e resultados.

Subcampo da inteligência artificial, o aprendizado de máquina é a área que engloba o estudo e a construção de sistemas inteligentes a partir de dados. (MOHRI, ROSTAMIZADEH e TALWALKAR, 2012) Após efetuado o aprendizado, também denominado treinamento, um sistema pode ser utilizado para classificar ou prever saídas para instâncias desconhecidas. (SIMON, 2013) A proposta deste trabalho consiste no estudo e aplicação de técnicas de aprendizado de máquina para determinar um algoritmo genérico de previsão das equações de chuvas intensas.
