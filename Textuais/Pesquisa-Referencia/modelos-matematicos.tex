\section{Modelos matemáticos que expressam a relação IDF}

As relações intensidade-duração-frequência podem ser expressas matematicamente, ou seja, equações propostas que representam a quantidade máxima de uma chuva. A definição do modelo que representa as chuvas intensas, utilizado em cada trabalho, deve ser de fácil manuseio, mantendo a segurança em seus resultados que devem ser o mais próximo à realidade (MELLO, LIMA e MELLO, 2003).

Segundo Villela e Mattos (1975), o modelo matemático clássico mais utilizado é expresso pela Equação 1.

Em situações onde a escassez de dados é muito grande, (MELLO, LIMA e MELLO, 2003) evidenciam a importância da comparação entre vários modelos que representam a relação IDF, podendo resultar em informações confiáveis e precisas. Em seu trabalho, os autores, utilizam modelos exponencial e linear (Equação 2 e 3).

Mello et al. (2003) concluíram que o modelo exponencial produziu erros menores, gerando melhores aproximações da intensidade máxima de chuvas. O modelo linear, apesar de possuir um coeficiente de determinação $r^2$ superior a 0,99, não foi considerado confiável para utilização em estudos de chuvas intensas pois apresenta os maiores erros médios e, também, a maior amplitude de erros comparado ao modelo exponencial.

Pfafstetter (1957), em seu trabalho de determinação das curvas IDF, realizou o ajuste do modelo representado pela Equação 4.

A Equação 4 fornece a precipitação para período de retorno de 1 ano e a Equação 5 permite estimar a chuva para outros tempos de retorno (TUCCI, 2004).

Bertoni e Tucci (2001) detalharam a metodologia de Bell, que relaciona a precipitação máxima para um tempo de duração definido e período de retorno a uma precipitação padrão de 60 min de duração e 2 anos de período de retorno (Equação 6).

Se acordo com Sampaio (2011), a metodologia Bell se baseia em séries de chuva, observados em todo globo terrestre, destacando que o valor máximo das chuvas está relacionando a regiões convectivas com características parecidas no mundo todo; assinalando isso com uma desvantagem do método, uma vez que as equações são geradas de valores médias e não específicos para uma determinada região. O autor, também, apontou como desvantagem que o valor da precipitação máxima obtida é válido apenas para durações entre 5 a 120 min.
Mello et al. (2003) comentaram que a principal característica do método de Bell é o ajuste da equação, que pode ser regionalizada e que alguns autores optam pelo emprego do modelo de Bell, para o Brasil, atribuindo valores fixos aos parâmetros de ajuste, variando apenas o período de retorno e a intensidade da chuva.
    
Mello et al. (2003) realizaram ajustes do modelo de Bell para as seguintes regiões: Norte, Sul, Centro, Leste e Triângulo Mineiro e alcançaram um desvio inferior a 8\% entre os valores empíricos e teóricos.

O método de Bell mostra-se adequado para estimar as máximas precipitações de curta durações, sendo uma opção na determinação das chuvas críticas de projeto quando as séries disponíveis contém poucos anos de observação (Oliveira, Antonini e Griebeler 2008). Oliveira et al. (2011) aplicaram o modelo de Bell para PCD no Estado do Mato Grosso e verificaram que, comparado com as relações IDF elaboradas pelo modelo clássico (apresentado na Equação 1), o modelo de Bell superestimou a chuva de projeto.

Chen (1983) sugeriu uma equação IDF utilizando três alturas de precipitação: chuva com duração de 1 hora e período de retorno de 10 anos; chuva com duração de 24 horas e período de retorno de 10 anos; chuva com duração de 1 hora e período de retorno de 100 anos. De acordo com o autor, verificou-se que, nas precipitações a partir da duração de 2 horas, as relações de duração em relação à chuva de 24 horas variaram em função da relação da chuva de 1 hora e à de 24 horas. A Equação 7, proposta por (CHEN, 1983), foi desenvolvida para as séries anuais.