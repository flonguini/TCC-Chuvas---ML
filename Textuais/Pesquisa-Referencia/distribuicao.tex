\section{Ajuste a distribuições estatísticas}

O desenvolvimento tecnológico e científico possibilita registrar o comportamento das variáveis hidrológicas, como por exemplo, chuvas, níveis de rios e nevascas. O acúmulo dos dados permitiu a criação de séries históricas, as quais são analisadas utilizando à estatística como uma ferramenta de tratamento de dados, dessa maneira o conhecimento dos conceitos estatísticos é indispensável ao desenvolvimento de estudos em hidrologia (PEIXINHO, 2007).

\textit{As variáveis hidrológicas e hidrometerológicas têm sua variabilidade registrada por meio das chamadas séries temporais, as quais reúnem as observações ou medições daquela variável, organizadas de forma sequencial de sua ocorrência no tempo (ou espaço). Por limitações impostas pelos processos de medição ou observações, as variáveis hidrológicas, embora apresentem variações instantâneas são contínuas ao longo do tempo, ou do espaço, têm seus registros separados por determinados intervalos de tempo, ou de distância (Naghettini e Pinto, 2007).}

De acordo com Naghettini e Pinto (2007) as séries hidrológicas podem incluir todas as observações disponíveis, coletadas em intervalos de tempo regulares ao longo de vários anos de registros, ou apenas alguns de seus valores característicos como, por exemplo, os máximos anuais ou as médias mensais. 

Com o ajuste de um modelo de chuva diária baseando0se nos estudos de Back (1997) é possível aumentar a eficiência dos dados de chuvas, principalmente por conseguir a simulação e criação de extensas séries de precipitação, sendo estas, muitas vezes, maiores que as próprias séries de dados observados.

No caso específico de eventos hidrológicos extremos, como por exemplo máximos e mínimos, as séries reduzidas podem ser anuais, quando os registros consecutivos possuem o mesmo intervalo no tempo, ou de duração segmentada, em caso contrário.
As séries de máximos valores são empregadas para ajuste, segundo a lei probabilística que melhor descreva o processo, possibilitando extrapolações (VIEIRA et al, 1991).

\textit{Distribuições teóricas de probabilidade são simplesmente funções analíticas usadas para descrever o comportamento de determinadas variáveis. No caso de extremos, só os ajustes das séries longas em múltiplas localidades é que dá indicações sobre as distribuições que levam a melhor extrapolação (SANSIGOLO, 2008).}

Embora a teoria probabilística fundamental de valores extremos tenha sido desenvolvida há muito tempo, a modelagem estatística de extremos ainda permanece como assunto ativo de pesquisas dado seu importante papel nos projetos e gerenciamento de recursos hídricos, especialmente em um contexto de mudanças climáticas (Katz et al, 2002).

A teoria de valores extremos é fundamental nestes casos para a modelagem destes eventos. Os fundamentos desta teoria foram desenvolvidos por Fisher- Tippett (1928), que definiram os três tipos possíveis de distribuições assintóticas de valores extremos, conhecidas como de Gumbel (tipo I), Fréchet (tipo II) e Weibull (tipo III) (Gumbel, 1958), que são casos especiais da Distribuição Generalizada de Valores Extremos desenvolvida por Jenkinson (1955). Além das distribuições de valores extremos, também são bastante utilizadas para descrever eventos raros as distribuições Log-normal e Pearson III (Sevruk e Geiger, 1981).

Diferentes distribuições, escolhidas entre as mais frequentemente utilizadas na descrição destas variáveis são consideradas, incluindo a de Gumbel, Log-Normal, Pearson III, Fréchet e Weibull, que tem, respectivamente, como funções de densidade de probabilidade acumulada.

\subsection{Distribuição de Gumbel}

Em geral, as distribuições de valores extremos de grandezas hidrológicas ajustam-se adequadamente à distribuição de Fisher- Tippett do tipo I, também nomeada como função de Gumbel ( VILLELA E MATTOS, 1975; LEOPOLDO et al,1984).

A distribuição de probabilidade de Gumbel é aplicada às séries históricas de valores extremos, especialmente, a precipitação máxima diária anual, sendo expressa pela Equação 10.

\subsection{Distribuição de Fréchet}

A distribuição de Fréchet é uma forma particular da distribuição de valores extremos do Tipo II e de acordo com Naghettini e Pinto (2007), essa distribuição é conhecida também pela denominação Log-Gumbel, sendo pela Equação 13

\textit{No caso dos valores máximos, a distribuição de Fréchet refere-se à forma assintótica limite para um conjunto de N variáveis aleatórias originais, independentes e igualmente distribuídas conforme um modelo, de cauda superior polinomial. A distribuição foi usada pela primeira vez na análise de frequência de vazões de enchentes por Fréchet (1927), tendo, desde então, encontrado aplicações, como distribuição extrema de eventos hidrológicos máximos.}

\subsection{Distribuição de Weibull}

Segundo Catalunha et al (2002) a distribuição de Weibull é utilizada em análise hidrológica para eventos extremos, sendo pouco conhecida a sua utilização em séries climáticas. Tem como principal método de ajuste de distribuição o da máxima verossimilhança, que consiste em determinar os valores de g e b pelas suas equações fundamentais.

A distribuição de Weibull a dois parâmetros tem como função de distribuição de probabilidades acumulada (FDA) a Equação 14.

Catalunha et al (2002) realizou um trabalho em Minas Gerais, concluindo que a distribuição de Weibull possui um ótimo desempenho para estimar as precipitações diárias.

Pfafstetter (1957), realizando os primeiros estudos no Brasil, utilizou séries de valores extremos de precipitações de 98 plataformas de coleta de dados distribuídas em várias regiões do Brasil, para a construção de curvas IDF, utilizando com ferramenta estatística a distribuição de Weibull.

\subsection{Distribuição Log-Normal}

A distribuição Log-Normal tem como função de distribuição de probabilidades acumulada (FDA) a Equação 15.

A distribuição Log-Normal mostrou-se adequada para previsão das chuvas prováveis apenas nos meses de maiores intensidades segundo Sampaio et al. (1999)