\section{Formação das precipitações}

De acordo com (MARENGO, 2008), a superfície terrestre está coberta em sua maior parte por água, este elemento representa 70\% da superfície da Terra estando sempre em constante movimento. A água é um elemento básico para a sobrevivência de todos os organismos vivos, pode-se encontrar a água nos três estados da matéria (sólido, líquido e gasoso), sendo possível observar as três fases no ciclo hidrológico (TUNDISI, 2003).

O ciclo hidrológico é o movimento contínuo da água presente nos oceanos, continentes e na atmosfera, é o grande responsável pela distribuição e disponibilidade de água no planeta. Os principais processos que envolvem o ciclo hidrológico são: precipitação, interceptação, evaporação, transpiração, infiltração, percolação e escoamento superficial (TUNDISI, 2003).

A definição do regime hidrológico ocorre pela combinação das características físicas de cada região (geologia, topografia e clima) e da ação do ciclo hidrológico. Assim sendo, existe uma desigualdade na distribuição de água, espacialmente e temporalmente, o que leva à necessidade de ações de planejamento ambiental de acordo com a situação de excesso ou escassez (TUNDISI, 2003).

A chuva pode ser definida como a precipitação de partículas de água líquida sob a forma de gotas de diâmetro superior a 0,5 mm (MACHADO e TORRES, 2012). As precipitações pluviais podem ser classificadas, conforme a sua origem, de acordo com o mecanismo de ascensão do ar úmido que proporciona a formação das nuvens, sendo que os principais tipos são: ciclônicas (frontais e não frontais), orográficas ou de relevo e convectivas ou de convecção (MIRANDA, OLIVEIRA e SILVA, 2010).

O encontro de massas de ar com propriedades diferentes originam as precipitações ciclônicas, sendo classificadas como frontais e não frontais. As precipitações não frontais podem ser geradas devido à queda de pressão, resultando na elevação do ar em razão da convergência horizontal em áreas de baixa pressão. As chuvas frontais ocorrem, quando a frente fria invade o local, empurrando, para cima, o ar quente e úmido, provocando resfriamento e condensação. São chuvas de grandes durações, abrangem grandes área e de intensidade média (TUCCI, 1993).

As precipitações orográficas (também nomeadas de chuvas de relevo), ocorrem durante a ascensão de massa de ar quente e úmido, pelo encontro de um obstáculo (serras são um exemplo) forçando a elevar-se e, consequentemente, reduzindo a temperatura sucedendo a condensação. Este tipo de chuva ocorre em pequenas áreas, sendo de baixa intensidade e extensa duração (MIRANDA, OLIVEIRA e SILVA, 2010). Após a ocorrência da precipitação, algumas vezes, a massa de ar consegue transpor a barreira, projetando a sombra pluviométrica, caracterizada por regiões secas, devido à umidade já ter sido, em grande parte, descarregada no lado oposto (FORGIARINI, VENDRUSCOLO e RIZZI, 2014).

\textit{Na classificação de precipitações convectivas, enquadram-se as chuvas intensas, típicas de regiões tropicais. A superfície aquecida, desigualmente, forma camadas de ar com densidades diferentes se mantendo em equilíbrio instável. Com a quebra desse equilíbrio (vento, superaquecimento), ocorre a ascensão brusca do ar menos denso, capaz de alcançar grandes altitudes, que atinge o nível de condensação e precipita (VILLELA e MATOS, 1975).}