\section{Análise de frequência de séries históricas}

\subsection{Tipos de séries}

Os projetos de drenagem urbana são projetados com período de retorno de 5, 10 ou mais anos, em média. Por esse motivo, é necessário o conhecimento da frequência de ocorrência dos eventos extremos.

As relações entre intensidade, duração e frequência das chuvas intensas são inferidas das observações de precipitações de um grande período de observações, para que seja aceitável as frequências como probabilidades. Essas relações se efetuarão em curvas de intensidade-duração, uma para cada frequência, todas com caráter de regularidade \cite{drenagem-superficial}.

\textit{Dois tipos de séries podem ser utilizados nas análises de frequências dos dados de chuva: as séries anuais que incluem a altura pluviométrica máxima de cada ano, e as séries parciais constituídas por alturas pluviométricas acima de um certo valor-base, independente do ano em que possam ocorrer \cite{drenagem-superficial}.}

A escolha do tipo da série depende do tamanho da mesma e do objetivo do estudo. As séries parciais fornecem resultados mais consistentes para períodos de retorno inferiores a 5 anos, e números de anos de dados menores que 12 anos \cite{tucci1993}.

Além disso, as duas séries contemplam, praticamente, os mesmos resultados para períodos de returno superiores a 10 anos \cite{manual-daee}.