\chapter{Resultados e Discussão}

Este capítulo apresenta os resultados dos parâmetros calculados para a Equação \ref{eq:equacao-geral}. Para isso, foi desenvolvido um código computacional, desenvolvido em Python 3.2, implementando a teoria apresentada nos capítulos anteriores. Também é apresentado o desempenho obtido pela arquitetura de rede neural artificial estudada e as curvas para cada período de retorno estudado, utilizando 1, 2, 5 e 10 neurônios.

\section{Resultados para o modelo estatístico}

Os registros da cidade de Recife foram colocados em sequência cronológica de segundos. Posteriormente, selecionaram-se as máximas precipitações anuais por duração, sempre verificando se estavam de acordo com os critérios estabelecidos no capítulo anterior. A Tabela \ref{tab:precipitacoes-recife} exibe as precipitações selecionadas para cada duração entre os anos de 2003 e 2011.

%precipitação máxima
\begin{table}[h]
\caption{Precipitações máximas para a cidade de Recife}
\begin{tabular}{ccccccccccc}
\toprule
\centering
Ano & \multicolumn{10}{c}{Duração (min)}                                                  \\ \cline{2-11} 
     & 5     & 10    & 15    & 30    & 60    & 120    & 240    & 360    & 720    & 1080   \\ \hline
2003 & 8     & 14,25 & 16,5  & 22,25 & 29,25 & 36,5   & 50,75  & 61,5   & 94,25  & 100,25 \\
2004 & 8     & 11,5  & 16,5  & 29,5  & 45,25 & 52,25  & 53,5   & 54     & 81     & 90,25  \\
2005 & 17,25 & 29,25 & 36,5  & 51,25 & 69,75 & 102,25 & 137,25 & 176,75 & 206,5  & 206,5  \\
2006 & 8,75  & 13,25 & 17,5  & 29    & 45,25 & 62,25  & 71,5   & 93,25  & 102,75 & 102,75 \\
2007 & 9,5   & 13    & 17,25 & 26,75 & 40,5  & 41,75  & 47,25  & 56     & 73,75  & 80,75  \\
2008 & 11    & 20,5  & 28    & 41,75 & 63    & 66,5   & 68,75  & 86     & 111,25 & 112,75 \\
2009 & 9,25  & 15,5  & 20,25 & 37,75 & 47,5  & 55,5   & 82     & 97,25  & 110,5  & 112,5  \\
2010 & 9,75  & 16,5  & 20,75 & 26    & 33,25 & 44,5   & 77,75  & 95     & 112    & 125    \\
2011 & 12,25 & 24    & 32,5  & 57    & 71,25 & 87,25  & 100,75 & 102    & 124,75 & 125,75 \\ \bottomrule
\end{tabular}
\fonte{Autores}
\label{tab:precipitacoes-recife}
\end{table}

%gumbel
As intensidades máximas anuais calculadas por meio da distribuição de probabilidade de Gumbel para máximos podem ser visualizadas na Tabela \ref{tab:gumbel}.


\begin{table}[h]
\centering
\caption{Intensidades em mm/h para distribuição de Gumbel}
\begin{tabular}{@{}cccccccccc@{}}
\toprule
\multirow{2}{*}{\begin{tabular}[c]{@{}l@{}}Duração\\ (min)\end{tabular}} & \multicolumn{9}{c}{Período de Retorno (anos)} \\ \cmidrule(l){2-10} 
  & 2      & 3      & 5      & 10     & 15     & 20     & 25     & 50     & 100    \\ \midrule
5 & 119,27 & 133,84 & 150,07 & 170,45 & 181,96 & 190,01 & 196,21 & 215,32 & 234,29 \\
10 & 99,34  & 114,16 & 130,65 & 151,39 & 163,08 & 171,27 & 177,58 & 197,01 & 216,30 \\
15 & 86,47  & 99,12  & 113,20 & 130,90 & 140,89 & 147,88 & 153,26 & 169,85 & 186,32 \\
30 & 67,41  & 77,54  & 88,82  & 102,99 & 110,99 & 116,59 & 120,90 & 134,19 & 147,38 \\
60 & 46,94  & 53,31  & 60,41  & 69,33  & 74,36  & 77,89  & 80,60  & 88,96  & 97,26  \\
120 & 28,70  & 33,24  & 38,29  & 44,64  & 48,23  & 50,74  & 52,67  & 58,62  & 64,53  \\
240 & 17,99  & 20,95  & 24,25  & 28,40  & 30,75  & 32,38  & 33,65  & 37,54  & 41,40  \\
360 & 14,20  & 16,79  & 19,67  & 23,28  & 25,33  & 26,76  & 27,86  & 31,25  & 34,62  \\
720 & 8,89   & 10,23  & 11,73  & 13,61  & 14,67  & 15,41  & 15,98  & 17,75  & 19,50  \\
1080 & 6,19   & 7,04   & 7,98   & 9,17   & 9,84   & 10,31  & 10,67  & 11,79  & 12,89  \\ \bottomrule
\end{tabular}
\fonte{Autores}
\label{tab:gumbel}
\end{table}

Apesar das séries de máximas precipitações anuais possuírem um intervalo de dados inferior ao recomendado pela literatura (de 30 anos), o resultado do Teste de Kolmogorov-Smirnov indica que essas amostras são representativas das precipitações extremas. Ainda que o Teste de Kolmogorov-Smirnov torna-se mais rigoroso para níveis de significância superiores a 20\%, não se pode rejeitar a hipótese de que a distribuição estatística escolhida (distribuição de Gumbel) representa as chuvas para a região em estudo.
O coeficiente de determinação ficou entre o intervalo de 0,89624 e 0,98455 para todas as durações estudadas. Os valores desses coeficientes legitimam a aderência da distribuição de Gumbel aos dados dessa região. A Tabela \ref{tab:teste-de-ks} exibe um resumo dos testes realizados. 

\begin{table}[h]
\centering
\caption{Resultados do teste de aderência para o teste de Kolmogorov-Smirnov}
\begin{tabular}{@{}cccccccc@{}}
\toprule
\multirow{2}{*}{\begin{tabular}[c]{@{}c@{}}Duração\\ (min)\end{tabular}} & \multirow{2}{*}{\begin{tabular}[c]{@{}c@{}}Divergência \\ Máxima\end{tabular}} & \multicolumn{4}{c}{\begin{tabular}[c]{@{}c@{}}Valores do teste de KS com n=9\\ Nível de significância:\end{tabular}} & \multirow{2}{*}{\begin{tabular}[c]{@{}c@{}}Passa no\\ Teste KS?\end{tabular}} & \multirow{2}{*}{Ajuste R2} \\ \cmidrule(lr){3-6}
             &      & 20\%    & 10\%    & 5\%     & 1\%    &     &        \\ \midrule
5            & 0,1289             & 0,3390  & 0,3880  & 0,4320  & 0,5140 & Sim & 0,93677      \\
10           & 0,1516             & 0,3390  & 0,3880  & 0,4320  & 0,5140 & Sim & 0,95283      \\
15           & 0,1605             & 0,3390  & 0,3880  & 0,4320  & 0,5140 & Sim & 0,89624      \\
30           & 0,1315             & 0,3390  & 0,3880  & 0,4320  & 0,5140 & Sim & 0,95010      \\
60           & 0,0880             & 0,3390  & 0,3880  & 0,4320  & 0,5140 & Sim & 0,95730      \\
120          & 0,0968             & 0,3390  & 0,3880  & 0,4320  & 0,5140 & Sim & 0,98455      \\
240          & 0,1213             & 0,3390  & 0,3880  & 0,4320  & 0,5140 & Sim & 0,97272      \\
360          & 0,1396             & 0,3390  & 0,3880  & 0,4320  & 0,5140 & Sim & 0,91924      \\
720          & 0,1418             & 0,3390  & 0,3880  & 0,4320  & 0,5140 & Sim & 0,93415      \\
1080         & 0,1033             & 0,3390  & 0,3880  & 0,4320  & 0,5140 & Sim & 0,92950 \\ \bottomrule
\end{tabular}
\fonte{Autores}
\label{tab:teste-de-ks}
\end{table}

Após a validação da distribuição estatística os valores dos coeficientes $k$, $a$, $b$ e $c$, foram determinados por meio de regressão linear múltipla. Os parâmetros finais para esses coeficientes são representados pela Equação \ref{eq:idf-recife}.

%eq chuva intensa
\begin{equation}
    i = \frac{1380,2176\times T_r^{0,19369}}{(t+22)^{0,78201}}
\label{eq:idf-recife}
\end{equation}

De posse dos coeficientes que melhor se ajustaram à distribuição escolhida é possível montar a família de curvas que representam a intensidade máxima de chuva para cada período de retorno em função da duração. A Figura \ref{fig:idf-ajustada} representa as oito curvas que correlacionam intensidade-duração-frequência para a cidade de Recife.

\begin{figure}[h]
    \caption{Curvas IDF ajustadas para a cidade de Recife}
    \centering
    \includegraphics[width=0.9\textwidth]{Textuais/Figuras/curvas-idf-ajustadas.png}
    \fonte{Autores}
    \label{fig:idf-ajustada}
\end{figure}

%rede
\section{Resultados para o modelo de inteligência artificial}

A Tabela \ref{tab:resumo} apresenta uma síntese dos coeficientes de determinação estudados nesse trabalho. Foram analisados os períodos de retorno entre 2 e 100 anos, com uma arquitetura de rede com 1, 2, 5 e 10 neurônios de ativação utilizando a função tangente hiperbólica como função de ativação.

\begin{table}[h]
\centering
\caption{Resumo dos coeficientes de determinação para a rede estudada}
\begin{tabular}{@{}ccccccccc@{}}
\toprule
\begin{tabular}[c]{@{}c@{}}Número de \end{tabular} & \multicolumn{8}{c}{Período de Retorno} \\ \cmidrule(l){2-9} 
Neurônios   & 2      & 5      & 10     & 15     & 20     & 25     & 50     & 100    \\ \midrule
1  & 0,9789 & 0,9687 & 0,9654 & 0,9754 & 0,9562 & 0,9658 & 0,9812 & 0,9612 \\
2  & 0,9892 & 0,9756 & 0,9658 & 0,9654 & 0,9654 & 0,9658 & 0,9881 & 0,9781 \\
5  & 0,9951 & 0,9878 & 0,9745 & 0,9851 & 0,9874 & 0,9781 & 0,9956 & 0,9785 \\
10 & 0,9967 & 0,9973 & 0,9856 & 0,9914 & 0,9881 & 0,9789 & 0,9961 & 0,9965 \\ \bottomrule
\end{tabular}
\label{tab:resumo}
\fonte{Autores}
\end{table}


As Figuras \ref{fig:tr2-1n} e \ref{fig:tr2-2n} apresentam os gráficos com a curva ajustada para o período de retorno de 2 anos e 1 ou 2 neurônios na rede.

%TR 2
\begin{figure}[H]
    \caption{Curva ajustada para os dados para TR 2 anos e 1 neurônio artificial}
    \centering
    \includegraphics[width=0.74\textwidth]{Textuais/Figuras/NN/tr2-1neuronio.png}
    \fonte{Autores}
    \label{fig:tr2-1n}
\end{figure}


\begin{figure}[H]
    \caption{Curva ajustada para os dados para TR 2 anos e 2 neurônios artificiais}
    \centering
    \includegraphics[width=0.74\textwidth]{Textuais/Figuras/NN/tr2-2neuronio.png}
    \fonte{Autores}
    \label{fig:tr2-2n}
\end{figure}

\begin{figure}[H]
    \caption{Curva ajustada para os dados para TR 2 anos e 5 neurônios artificiais}
    \centering
    \includegraphics[width=0.74\textwidth]{Textuais/Figuras/NN/tr2-5neuronio.png}
    \fonte{Autores}
    \label{fig:tr2-5n}
\end{figure}

\begin{figure}[H]
    \caption{Curva ajustada para os dados para TR 2 anos e 10 neurônios artificiais}
    \centering
    \includegraphics[width=0.74\textwidth]{Textuais/Figuras/NN/tr2-10neuronio.png}
    \fonte{Autores}
    \label{fig:tr2-10n}
\end{figure}
%FIM TR 2

\begin{figure}[H]
    \caption{Comparação entre as curvas geradas para TR2 e 10 neurônios}
    \centering
    \includegraphics[width=\textwidth]{Textuais/Resultados/Comparacao/TR2.png}
    \fonte{Autores}
    \label{fig:comp-tr2}
\end{figure}

%TR 3
\begin{figure}[H]
    \caption{Curva ajustada para os dados para TR 3 anos e 1 neurônio artificial}
    \centering
    \includegraphics[width=0.74\textwidth]{Textuais/Figuras/NN/tr3-1neuronio.png}
    \fonte{Autores}
    \label{fig:tr3-1n}
\end{figure}

\begin{figure}[H]
    \caption{Curva ajustada para os dados para TR 3 anos e 2 neurônios artificiais}
    \centering
    \includegraphics[width=0.74\textwidth]{Textuais/Figuras/NN/tr3-2neuronio.png}
    \fonte{Autores}
    \label{fig:tr3-2n}
\end{figure}

\begin{figure}[H]
    \caption{Curva ajustada para os dados para TR 3 anos e 5 neurônios artificiais}
    \centering
    \includegraphics[width=0.74\textwidth]{Textuais/Figuras/NN/tr3-5neuronio.png}
    \fonte{Autores}
    \label{fig:tr3-5n}
\end{figure}

\begin{figure}[H]
    \caption{Curva ajustada para os dados para TR 3 anos e 10 neurônios artificiais}
    \centering
    \includegraphics[width=0.74\textwidth]{Textuais/Figuras/NN/tr3-10neuronio.png}
    \fonte{Autores}
    \label{fig:tr3-10n}
\end{figure}
%FIM TR 3

\begin{figure}[H]
    \caption{Comparação entre as curvas geradas para TR3 e 10 neurônios}
    \centering
    \includegraphics[width=\textwidth]{Textuais/Resultados/Comparacao/TR3.png}
    \fonte{Autores}
    \label{fig:comp-tr3}
\end{figure}
%TR 5
\begin{figure}[H]
    \caption{Curva ajustada para os dados para TR 5 anos e 1 neurônio artificial}
    \centering
    \includegraphics[width=0.74\textwidth]{Textuais/Figuras/NN/tr5-1neuronio.png}
    \fonte{Autores}
    \label{fig:tr5-1n}
\end{figure}

\begin{figure}[H]
    \caption{Curva ajustada para os dados para TR 5 anos e 2 neurônios artificiais}
    \centering
    \includegraphics[width=0.74\textwidth]{Textuais/Figuras/NN/tr5-2neuronio.png}
    \fonte{Autores}
    \label{fig:tr5-2n}
\end{figure}

\begin{figure}[H]
    \caption{Curva ajustada para os dados para TR 5 anos e 5 neurônios artificiais}
    \centering
    \includegraphics[width=0.74\textwidth]{Textuais/Figuras/NN/tr5-5neuronio.png}
    \fonte{Autores}
    \label{fig:tr5-5n}
\end{figure}

\begin{figure}[H]
    \caption{Curva ajustada para os dados para TR 5 anos e 10 neurônios artificiais}
    \centering
    \includegraphics[width=0.74\textwidth]{Textuais/Figuras/NN/tr5-10neuronio.png}
    \fonte{Autores}
    \label{fig:tr5-10n}
\end{figure}
%FIM TR 5

\begin{figure}[H]
    \caption{Comparação entre as curvas geradas para TR5 e 10 neurônios}
    \centering
    \includegraphics[width=\textwidth]{Textuais/Resultados/Comparacao/TR5.png}
    \fonte{Autores}
    \label{fig:comp-tr5}
\end{figure}
%TR 10
\begin{figure}[H]
    \caption{Curva ajustada para os dados para TR 10 anos e 1 neurônio artificial}
    \centering
    \includegraphics[width=0.74\textwidth]{Textuais/Figuras/NN/tr10-1neuronio.png}
    \fonte{Autores}
    \label{fig:tr10-1n}
\end{figure}

\begin{figure}[H]
    \caption{Curva ajustada para os dados para TR 10 anos e 2 neurônios artificiais}
    \centering
    \includegraphics[width=0.74\textwidth]{Textuais/Figuras/NN/tr10-2neuronio.png}
    \fonte{Autores}
    \label{fig:tr10-2n}
\end{figure}

\begin{figure}[H]
    \caption{Curva ajustada para os dados para TR 10 anos e 5 neurônios artificiais}
    \centering
    \includegraphics[width=0.74\textwidth]{Textuais/Figuras/NN/tr10-5neuronio.png}
    \fonte{Autores}
    \label{fig:tr10-5n}
\end{figure}

\begin{figure}[H]
    \caption{Curva ajustada para os dados para TR 10 anos e 10 neurônios artificiais}
    \centering
    \includegraphics[width=0.74\textwidth]{Textuais/Figuras/NN/tr10-10neuronio.png}
    \fonte{Autores}
    \label{fig:tr10-10n}
\end{figure}
%FIM TR 10

\begin{figure}[H]
    \caption{Comparação entre as curvas geradas para TR10 e 10 neurônios}
    \centering
    \includegraphics[width=\textwidth]{Textuais/Resultados/Comparacao/TR10.png}
    \fonte{Autores}
    \label{fig:comp-tr10}
\end{figure}
%TR 15
\begin{figure}[H]
    \caption{Curva ajustada para os dados para TR 15 anos e 1 neurônio artificial}
    \centering
    \includegraphics[width=0.74\textwidth]{Textuais/Figuras/NN/tr15-1neuronio.png}
    \fonte{Autores}
    \label{fig:tr15-1n}
\end{figure}

\begin{figure}[H]
    \caption{Curva ajustada para os dados para TR 15 anos e 2 neurônios artificiais}
    \centering
    \includegraphics[width=0.74\textwidth]{Textuais/Figuras/NN/tr15-2neuronio.png}
    \fonte{Autores}
    \label{fig:tr15-2n}
\end{figure}

\begin{figure}[H]
    \caption{Curva ajustada para os dados para TR 15 anos e 5 neurônios artificiais}
    \centering
    \includegraphics[width=0.74\textwidth]{Textuais/Figuras/NN/tr15-5neuronio.png}
    \fonte{Autores}
    \label{fig:tr15-5n}
\end{figure}

\begin{figure}[H]
    \caption{Curva ajustada para os dados para TR 15 anos e 10 neurônios artificiais}
    \centering
    \includegraphics[width=0.74\textwidth]{Textuais/Figuras/NN/tr15-10neuronio.png}
    \fonte{Autores}
    \label{fig:tr15-10n}
\end{figure}
%FIM TR 15

\begin{figure}[H]
    \caption{Comparação entre as curvas geradas para TR15 e 10 neurônios}
    \centering
    \includegraphics[width=\textwidth]{Textuais/Resultados/Comparacao/TR15.png}
    \fonte{Autores}
    \label{fig:comp-tr15}
\end{figure}
%TR 20
\begin{figure}[H]
    \caption{Curva ajustada para os dados para TR 20 anos e 1 neurônio artificial}
    \centering
    \includegraphics[width=0.74\textwidth]{Textuais/Figuras/NN/tr20-1neuronio.png}
    \fonte{Autores}
    \label{fig:tr20-1n}
\end{figure}

\begin{figure}[H]
    \caption{Curva ajustada para os dados para TR 20 anos e 2 neurônios artificiais}
    \centering
    \includegraphics[width=0.74\textwidth]{Textuais/Figuras/NN/tr20-2neuronio.png}
    \fonte{Autores}
    \label{fig:tr20-2n}
\end{figure}

\begin{figure}[H]
    \caption{Curva ajustada para os dados para TR 20 anos e 5 neurônios artificiais}
    \centering
    \includegraphics[width=0.74\textwidth]{Textuais/Figuras/NN/tr20-5neuronio.png}
    \fonte{Autores}
    \label{fig:tr20-5n}
\end{figure}

\begin{figure}[H]
    \caption{Curva ajustada para os dados para TR 20 anos e 10 neurônios artificiais}
    \centering
    \includegraphics[width=0.74\textwidth]{Textuais/Figuras/NN/tr20-10neuronio.png}
    \fonte{Autores}
    \label{fig:tr20-10n}
\end{figure}
%FIM TR 20

\begin{figure}[H]
    \caption{Comparação entre as curvas geradas para TR20 e 10 neurônios}
    \centering
    \includegraphics[width=\textwidth]{Textuais/Resultados/Comparacao/TR20.png}
    \fonte{Autores}
    \label{fig:comp-tr20}
\end{figure}
%TR 25
\begin{figure}[H]
    \caption{Curva ajustada para os dados para TR 25 anos e 1 neurônio artificial}
    \centering
    \includegraphics[width=0.74\textwidth]{Textuais/Figuras/NN/tr25-1neuronio.png}
    \fonte{Autores}
    \label{fig:tr25-1n}
\end{figure}

\begin{figure}[H]
    \caption{Curva ajustada para os dados para TR 25 anos e 2 neurônios artificiais}
    \centering
    \includegraphics[width=0.74\textwidth]{Textuais/Figuras/NN/tr25-2neuronio.png}
    \fonte{Autores}
    \label{fig:tr25-2n}
\end{figure}

\begin{figure}[H]
    \caption{Curva ajustada para os dados para TR 25 anos e 5 neurônios artificiais}
    \centering
    \includegraphics[width=0.74\textwidth]{Textuais/Figuras/NN/tr25-5neuronio.png}
    \fonte{Autores}
    \label{fig:tr25-5n}
\end{figure}

\begin{figure}[H]
    \caption{Curva ajustada para os dados para TR 25 anos e 10 neurônios artificiais}
    \centering
    \includegraphics[width=0.74\textwidth]{Textuais/Figuras/NN/tr25-10neuronio.png}
    \fonte{Autores}
    \label{fig:tr25-10n}
\end{figure}
%FIM TR 25

\begin{figure}[H]
    \caption{Comparação entre as curvas geradas para TR25 e 10 neurônios}
    \centering
    \includegraphics[width=\textwidth]{Textuais/Resultados/Comparacao/TR25.png}
    \fonte{Autores}
    \label{fig:comp-tr25}
\end{figure}
%TR 50
\begin{figure}[H]
    \caption{Curva ajustada para os dados para TR 50 anos e 1 neurônio artificial}
    \centering
    \includegraphics[width=0.74\textwidth]{Textuais/Figuras/NN/tr50-1neuronio.png}
    \fonte{Autores}
    \label{fig:tr50-1n}
\end{figure}

\begin{figure}[H]
    \caption{Curva ajustada para os dados para TR 50 anos e 2 neurônios artificiais}
    \centering
    \includegraphics[width=0.74\textwidth]{Textuais/Figuras/NN/tr50-2neuronio.png}
    \fonte{Autores}
    \label{fig:tr50-2n}
\end{figure}

\begin{figure}[H]
    \caption{Curva ajustada para os dados para TR 50 anos e 5 neurônios artificiais}
    \centering
    \includegraphics[width=0.74\textwidth]{Textuais/Figuras/NN/tr50-5neuronio.png}
    \fonte{Autores}
    \label{fig:tr50-5n}
\end{figure}

\begin{figure}[H]
    \caption{Curva ajustada para os dados para TR 50 anos e 10 neurônios artificiais}
    \centering
    \includegraphics[width=0.74\textwidth]{Textuais/Figuras/NN/tr50-10neuronio.png}
    \fonte{Autores}
    \label{fig:tr50-10n}
\end{figure}
%FIM TR 50

\begin{figure}[H]
    \caption{Comparação entre as curvas geradas para TR50 e 10 neurônios}
    \centering
    \includegraphics[width=\textwidth]{Textuais/Resultados/Comparacao/TR50.png}
    \fonte{Autores}
    \label{fig:comp-tr50}
\end{figure}
%TR 100
\begin{figure}[H]
    \caption{Curva ajustada para os dados para TR 100 anos e 1 neurônio artificial}
    \centering
    \includegraphics[width=0.74\textwidth]{Textuais/Figuras/NN/tr100-1neuronio.png}
    \fonte{Autores}
    \label{fig:tr100-1n}
\end{figure}

\begin{figure}[H]
    \caption{Curva ajustada para os dados para TR 100 anos e 2 neurônios artificiais}
    \centering
    \includegraphics[width=0.74\textwidth]{Textuais/Figuras/NN/tr100-2neuronio.png}
    \fonte{Autores}
    \label{fig:tr100-2n}
\end{figure}

\begin{figure}[H]
    \caption{Curva ajustada para os dados para TR 100 anos e 5 neurônios artificiais}
    \centering
    \includegraphics[width=0.74\textwidth]{Textuais/Figuras/NN/tr100-5neuronio.png}
    \fonte{Autores}
    \label{fig:tr100-5n}
\end{figure}

\begin{figure}[H]
    \caption{Curva ajustada para os dados para TR 100 anos e 10 neurônios artificiais}
    \centering
    \includegraphics[width=0.74\textwidth]{Textuais/Figuras/NN/tr100-10neuronio.png}
    \fonte{Autores}
    \label{fig:tr100-10n}
\end{figure}
%FIM TR 100

\begin{figure}[H]
    \caption{Comparação entre as curvas geradas para TR100 e 10 neurônios}
    \centering
    \includegraphics[width=\textwidth]{Textuais/Resultados/Comparacao/TR100.png}
    \fonte{Autores}
    \label{fig:comp-tr100}
\end{figure}

\subsection{Discussão}

%okay
\subsubsection{Análise para período de retorno de 2 anos}
A curva gerada pela RNA composta por 1 neurônio artificial apresentou erros iniciais de 25\% e ao término do período de aprendizagem esse erro foi reduzido para 3\%. Quando a rede é composta por 2 neurônios o erro inicial é reduzido para 13\% e converge para 2\%, mostrando que o número de neurônios tem um grande impacto na forma como a RNA irá se comportar. A utilização de 5 neurônios fez com que o erro inicial fosse bastante inferior aos anteriores, partindo de 1\%, porém ocorre uma grande oscilação até convergir para o valor de 0.125\% de erro. A RNA composta por 10 neurônios comporta-se de forma análoga a rede com 5 neurônios, apesar de convergir mais rapidamente e gerar um erro final inferior a 0.1\%. A curva final, gerada pela RNA com 10 neurônios, apresentam valores muito próximos da equação de Vilella-Mattos (1975), os valores que menos se aproximam estão entre o intervalo de 50 a 400 minutos com uma diferença de até 14,6\%.

\subsubsection{Análise para período de retorno de 3 anos}
Analisando a curva gerada para o período de retorno de 3 anos, a rede neural tem um comportamento semelhante ao analisado anteriormente. Quando a rede possui apenas um neurônio artificial o valor do erro estimado inicia-se com 30\% e, após o período de aprendizagem, converge para um erro de 5\%. Ao fazer o treinamento da rede utilizando dois neurônios, diferentemente dos valores para TR de 2 anos, houve uma convergência muito mais rápida, o erro inicial estimado foi de 13\%, porém ao término do treinamento o erro foi reduzido a 1\%, que já é um valor satisfatório. Para a RNA com 5 e 10 neurônios o comportamento foi bem parecido, o erro inicial está próximo de 1\% e após o treinamento o valor do erro fica próximo de 0.1\%. Comparando as curvas geradas pela RNA com 10 neurônios e a Equação de Vilella-Mattos (1975), apesar dos valores estarem muito próximos, os valores de intensidade mais distantes estão entre as durações de 50 e 230 minutos, apresentando uma diferença máxima de 8.63\%.

\subsubsection{Análise para período de retorno de 5 anos}
Para o período de retorno de 5 anos a RNA comportou-se semelhante ao período de retorno de 2 anos. A rede composta por 1 e 2 neurônios apresentam erros iniciais de aproximadamente 30\% e convergem para valores próximos de 5\%. Quando a rede é composta por 5 e 10 neurônios, além do erro inicial estar entre 2\% e 5\% é possível verificar que na metade do treinamento os valores já haviam convergido, o erro estimado para a rede com 5 e 10 neurônios é de 0.1\%. Ao comparar a curva gerada pela RNA de 10 neurônios com os valores da Equação de Vilella-Mattos (1975), verifica-se que os valores mais divergentes estão entre as durações de 230 a 700 minutos, apresentando uma diferença máxima de 7.23\%.

\subsubsection{Análise para período de retorno de 10 anos}
A saída de dados para o período de retorno de 10 anos e 1 neurônio artificial seguiu o padrão apresentado pelas outras análises, o erro estimado inicia-se em 40\% e após o treinamento da rede esse valor converge para 7\%. Ao fazer o treinamento da RNA com 2, 5 e 10 neurônios artificiais o comportamento foi análogo ao apresentado no TR3, os valores convergiram rapidamente apresentado erros na ordem de 0.1\%, porém diferente dos resultados para o período de retorno de 3 anos, os valores convergiram mais rápido quando a RNA foi treinada com 5 neurônios. Analisando as curvas da RNA com 5 neurônios e a curva gerada pela Equação de Vilella-Mattos (1975), pode-se observar que a maior divergência de valores está entre o intervalo de 300 e 700 minutos, apresentando uma diferença máxima de 8.23\%.

\subsubsection{Análise para período de retorno de 15 anos}
Analisando o comportamento da RNA para o período de retorno de 15 anos, quando a rede é composta por 1 neurônio artificial o erro inicial é de 38\%, após o treinamento da rede esse valor converge para 10\%, portanto apresentando valores não confiáveis devido ao grande erro. Ao criar a rede com 2, 5 e 10 neurônios o comportamento é bastante diferente, pois ocorre uma oscilação dos valores durante o período de treinamento, com erros iniciais de 10\%, e terminando com erros na ordem de 0.1\%. Os valores para 2, 5 e 10 neurônios foram muito próximos, portanto recomenda-se a utilização de 2 neurônios nesse caso para evitar processamento computacional desnecessário. Ao comparar a curva gerada pela RNA e a Equação de Vilella-Mattos (1975), observa-se que o maior erro está entre as durações de 220 minutos e 800 minutos, apresentando uma diferença máxima de 6.23\%.

\subsubsection{Análise para período de retorno de 20 anos}
O comportamento da RNA para o período de retorno de 20 anos com 1 e 2 neurônios, assim como para outros períodos de retorno, apresentam grandes erros iniciais e após o treinamento. Inicialmente esse erro está entre 20\% e 40\%, após o treinamento esse valor converge para 10\%. Quando se utiliza 5 e 10 neurônios o comportamento é bem semelhante, apresentam erros iniciais na ordem de 5\% e após o treinamento da rede convergem para valores próximos de 0.1\%, após uma oscilação no treinamento. Comparando as curvas geradas pela RNA com 10 neurônios e a curva gerada pela Equação de Vilella-Mattos (1975), a maior diferença entre os valores estão entre as durações de 220 minutos e 800 minutos, possuindo uma diferença máxima de 6.73\% em as duas curvas.

\subsubsection{Análise para período de retorno de 25 anos}
Quando se verifica o comportamento da RNA para o período de retorno de 25 anos com 1 e 2 neurônios artificiais, o erro inicial fica entre 40\% e 100\% e após o treinamento esse erro converge para 13\%. O comportamento da rede para 5 e 10 neurônios é bem semelhante, apresentando um erro inicial na ordem de 10\% e após algumas oscilações no treinamento convergem para valores próximos de 0.1\%. Comparando as curvas geradas pela RNA e pela Equação de Vilella-Mattos (1975), o maior erro está no intervalo de 220 e 800 minutos, com uma diferença máxima de 7.39\%.

\subsubsection{Análise para período de retorno de 50 anos}
Analisando o comportamento da RNA para período de retorno de 50 anos, assim como os outros resultados, quando a rede apresenta apenas um neurônio artificial os erros iniciais e finais são excessivos, ao iniciar o treinamento o erro é de 60\% e converge para 10\% após o treinamento. Quando se utiliza 2, 5 e 10 neurônios a rede se comporta de maneira parecida, o erro inicial está na ordem de 10\% e, após oscilações no período de treinamento, converge para valores inferiores a 0.1\%.  Ao comparar as duas curvas geradas, pela RNA e pela Equação de Vilella-Mattos (1975), a maior diferença entre os valores está no intervalo de 50 a 120 minutos, com um erro de 7.56\%.

\subsubsection{Análise para período de retorno de 100 anos}
Quando o período retorno analisado é de 100 anos, a RNA criada nesse trabalho, tem um comportamento análogo ao período de retorno de 25 anos. Ao fazer o treinamento com apenas um neurônio artificial o erro inicial está na ordem de 100\% e após o treinamento converge para 15\%. O desempenho da rede para 2, 5 e 10 neurônios são bem semelhantes, o erro inicial está na ordem de 10\%, após várias oscilações no treinamento esse erro converge para valores próximos de 0.1\%. Ao comparar as duas curvas geradas, pela RNA e pela Equação de Vilella-Mattos (1975), a maior diferença entre os valores está no intervalo de 30 a 200 minutos, com um erro máximo de 11.56\%.
