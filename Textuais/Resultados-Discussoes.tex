\chapter{Resultados e Discussão}

Texto da seção primária – Resultados e Discussão -- é o último texto dessa seção a ser produzido, pois deve ser seu RESUMO, indicando rapidamente como foram obtidos os resultados e qual foi a abordagem para a sua interpretação.

\section{Características dessa seção}

Com certeza, é a seção mais importante do trabalho, a que evidencia a capacidade de análise e síntese do autor, ou seja, sua capacidade de interpretar dados e informações, relacionando o que foi feito com o que já é conhecido.

Nessa seção são apresentadas as medidas realizadas devidamente tabuladas e apresentadas em gráficos. Estatísticas são feitas e discutidas para determinar relações de causa de efeito entre variáveis de entrada e saída. A capacidade dissertativa dos autores, de conseguir organizar e trabalhar as informações colhidas, relacionando as observações realizadas com conclusões devidamente fundamentadas deve ser mostrada nessa seção.

Com base nos resultados é comum encontrar na literatura textos que confirmem ou não as observações, caso sejam muito diferentes do que se conhece é que é aceito pela comunidade científica. É muito importante que existam citações de outros trabalhos na discussão dos resultados e suas as devidas comparações, pois isso evidencia o cuidado do autor para situar seu trabalho no campo de estudos sobre o tema e qual foi a contribuição do TCC nesse universo.