\subsection{Discussão}

%okay
\subsubsection{Análise para período de retorno de 2 anos}
A curva gerada pela RNA composta por 1 neurônio artificial apresentou erros iniciais de 25\% e ao término do período de aprendizagem esse erro foi reduzido para 3\%. Quando a rede é composta por 2 neurônios o erro inicial é reduzido para 13\% e converge para 2\%, mostrando que o número de neurônios tem um grande impacto na forma como a RNA irá se comportar. A utilização de 5 neurônios fez com que o erro inicial fosse bastante inferior aos anteriores, partindo de 1\%, porém ocorre uma grande oscilação até convergir para o valor de 0.125\% de erro. A RNA composta por 10 neurônios comporta-se de forma análoga a rede com 5 neurônios, apesar de convergir mais rapidamente e gerar um erro final inferior a 0.1\%. A curva final, gerada pela RNA com 10 neurônios, apresentam valores muito próximos da equação de Vilella-Mattos (1975), os valores que menos se aproximam estão entre o intervalo de 50 a 400 minutos com uma diferença de até 14,6\%.

\subsubsection{Análise para período de retorno de 3 anos}
Analisando a curva gerada para o período de retorno de 3 anos, a rede neural tem um comportamento semelhante ao analisado anteriormente. Quando a rede possui apenas um neurônio artificial o valor do erro estimado inicia-se com 30\% e, após o período de aprendizagem, converge para um erro de 5\%. Ao fazer o treinamento da rede utilizando dois neurônios, diferentemente dos valores para TR de 2 anos, houve uma convergência muito mais rápida, o erro inicial estimado foi de 13\%, porém ao término do treinamento o erro foi reduzido a 1\%, que já é um valor satisfatório. Para a RNA com 5 e 10 neurônios o comportamento foi bem parecido, o erro inicial está próximo de 1\% e após o treinamento o valor do erro fica próximo de 0.1\%. Comparando as curvas geradas pela RNA com 10 neurônios e a Equação de Vilella-Mattos (1975), apesar dos valores estarem muito próximos, os valores de intensidade mais distantes estão entre as durações de 50 e 230 minutos, apresentando uma diferença máxima de 8.63\%.

\subsubsection{Análise para período de retorno de 5 anos}
Para o período de retorno de 5 anos a RNA comportou-se semelhante ao período de retorno de 2 anos. A rede composta por 1 e 2 neurônios apresentam erros iniciais de aproximadamente 30\% e convergem para valores próximos de 5\%. Quando a rede é composta por 5 e 10 neurônios, além do erro inicial estar entre 2\% e 5\% é possível verificar que na metade do treinamento os valores já haviam convergido, o erro estimado para a rede com 5 e 10 neurônios é de 0.1\%. Ao comparar a curva gerada pela RNA de 10 neurônios com os valores da Equação de Vilella-Mattos (1975), verifica-se que os valores mais divergentes estão entre as durações de 230 a 700 minutos, apresentando uma diferença máxima de 7.23\%.

\subsubsection{Análise para período de retorno de 10 anos}
A saída de dados para o período de retorno de 10 anos e 1 neurônio artificial seguiu o padrão apresentado pelas outras análises, o erro estimado inicia-se em 40\% e após o treinamento da rede esse valor converge para 7\%. Ao fazer o treinamento da RNA com 2, 5 e 10 neurônios artificiais o comportamento foi análogo ao apresentado no TR3, os valores convergiram rapidamente apresentado erros na ordem de 0.1\%, porém diferente dos resultados para o período de retorno de 3 anos, os valores convergiram mais rápido quando a RNA foi treinada com 5 neurônios. Analisando as curvas da RNA com 5 neurônios e a curva gerada pela Equação de Vilella-Mattos (1975), pode-se observar que a maior divergência de valores está entre o intervalo de 300 e 700 minutos, apresentando uma diferença máxima de 8.23\%.

\subsubsection{Análise para período de retorno de 15 anos}
Analisando o comportamento da RNA para o período de retorno de 15 anos, quando a rede é composta por 1 neurônio artificial o erro inicial é de 38\%, após o treinamento da rede esse valor converge para 10\%, portanto apresentando valores não confiáveis devido ao grande erro. Ao criar a rede com 2, 5 e 10 neurônios o comportamento é bastante diferente, pois ocorre uma oscilação dos valores durante o período de treinamento, com erros iniciais de 10\%, e terminando com erros na ordem de 0.1\%. Os valores para 2, 5 e 10 neurônios foram muito próximos, portanto recomenda-se a utilização de 2 neurônios nesse caso para evitar processamento computacional desnecessário. Ao comparar a curva gerada pela RNA e a Equação de Vilella-Mattos (1975), observa-se que o maior erro está entre as durações de 220 minutos e 800 minutos, apresentando uma diferença máxima de 6.23\%.

\subsubsection{Análise para período de retorno de 20 anos}
O comportamento da RNA para o período de retorno de 20 anos com 1 e 2 neurônios, assim como para outros períodos de retorno, apresentam grandes erros iniciais e após o treinamento. Inicialmente esse erro está entre 20\% e 40\%, após o treinamento esse valor converge para 10\%. Quando se utiliza 5 e 10 neurônios o comportamento é bem semelhante, apresentam erros iniciais na ordem de 5\% e após o treinamento da rede convergem para valores próximos de 0.1\%, após uma oscilação no treinamento. Comparando as curvas geradas pela RNA com 10 neurônios e a curva gerada pela Equação de Vilella-Mattos (1975), a maior diferença entre os valores estão entre as durações de 220 minutos e 800 minutos, possuindo uma diferença máxima de 6.73\% em as duas curvas.

\subsubsection{Análise para período de retorno de 25 anos}
Quando se verifica o comportamento da RNA para o período de retorno de 25 anos com 1 e 2 neurônios artificiais, o erro inicial fica entre 40\% e 100\% e após o treinamento esse erro converge para 13\%. O comportamento da rede para 5 e 10 neurônios é bem semelhante, apresentando um erro inicial na ordem de 10\% e após algumas oscilações no treinamento convergem para valores próximos de 0.1\%. Comparando as curvas geradas pela RNA e pela Equação de Vilella-Mattos (1975), o maior erro está no intervalo de 220 e 800 minutos, com uma diferença máxima de 7.39\%.

\subsubsection{Análise para período de retorno de 50 anos}
Analisando o comportamento da RNA para período de retorno de 50 anos, assim como os outros resultados, quando a rede apresenta apenas um neurônio artificial os erros iniciais e finais são excessivos, ao iniciar o treinamento o erro é de 60\% e converge para 10\% após o treinamento. Quando se utiliza 2, 5 e 10 neurônios a rede se comporta de maneira parecida, o erro inicial está na ordem de 10\% e, após oscilações no período de treinamento, converge para valores inferiores a 0.1\%.  Ao comparar as duas curvas geradas, pela RNA e pela Equação de Vilella-Mattos (1975), a maior diferença entre os valores está no intervalo de 50 a 120 minutos, com um erro de 7.56\%.

\subsubsection{Análise para período de retorno de 100 anos}
Quando o período retorno analisado é de 100 anos, a RNA criada nesse trabalho, tem um comportamento análogo ao período de retorno de 25 anos. Ao fazer o treinamento com apenas um neurônio artificial o erro inicial está na ordem de 100\% e após o treinamento converge para 15\%. O desempenho da rede para 2, 5 e 10 neurônios são bem semelhantes, o erro inicial está na ordem de 10\%, após várias oscilações no treinamento esse erro converge para valores próximos de 0.1\%. Ao comparar as duas curvas geradas, pela RNA e pela Equação de Vilella-Mattos (1975), a maior diferença entre os valores está no intervalo de 30 a 200 minutos, com um erro máximo de 11.56\%.
