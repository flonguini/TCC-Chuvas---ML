\chapter{Conclusão}

Este trabalho apresentou uma proposta de rede neural artificial recorrente que consegue produzir as curvas de intensidade-duração-frequência para durações entre 5 e 1080 minutos, gerando as curvas para os períodos de retorno de 2, 3, 5, 15, 20, 25, 50 e 100 anos para a cidade de Recife. A série histórica utilizou dados coletados entre os anos 2003 e 2011, pois não apresentavam falhas nas medições. A distribuição de probabilidade de Gumbel mostrou-se representativa para distribuição empírica de probabilidade de eventos extremos para a região escolhida nesse trabalho, comprovada pela aplicação e aprovação no teste estatístico de Kolmogorov-Smirnov a níveis de significância de 1\% e 20\%.

Foi desenvolvido em Matlab a arquitetura da rede neural recorrente utilizando funções de ativação do tipo tangente hiperbólica e algoritmo de Levenberg–Marquardt como método de otimização. Como parâmetro de entrada utilizou-se as durações, e valores de saída as intensidades geradas pela distribuição de probabilidade de Gumbel. Para avaliar a capacidade de aprendizado da rede analisaram-se as saídas para redes compostas por 1, 2, 5 e 10 neurônios de ativação. As seguintes conclusões podem ser destacadas:

\begin{itemize}
    \item A rede neural artificial apresentou coeficientes de determinação entre 0,9789 e 0,9973 para os períodos de retorno entre 2 e 100 anos e durações entre 5 e 1080 minutos, mostrando-se eficiente no caso estudado nesse trabalho.
    \item Os resultados que apresentaram maiores coeficientes de determinação foram para as arquiteturas com 10 neurônios artificiais, atingindo valores de até 0,9973 para o período de retorno de 5 anos.
    \item Até o estudo desse trabalho, a utilização de redes neurais artificiais mostraram-se eficientes para determinação das curvas IDFs.
    \item Pode-se concluir que o número de neurônios impactam no erro final, a medida em que são acrescentados mais neurônios o erro, ao término do aprendizado, é cada vez menor. As RNAs estudadas nesse trabalho, composta por 5 e 10 neurônios, já apresentam desvios inferiores a 0.2\% em comparação as curvas geradas pela Equação de Villela-Mattos (1975). O acréscimo de mais neurônios pode gerar apenas um acréscimo de processamento computacional desnecessário.
\end{itemize}

\section{Trabalhos futuros}

Para dar prosseguimento ao trabalho aqui iniciado, são sugeridos os seguintes estudos futuros:

\begin{enumerate}
    \item Verificar se a rede neural consegue convergir quando as séries históricas contém falhas.
    \item Alterar os parâmetros da rede imputados para verificar a ocorrência de uma possível otimização.
    \item Modificar o algoritmo de convergência para avaliar a forma como a rede irá convergir.
    \item Testar a rede com uma série histórica superior a 30 anos.
    \item Verificar os resultados para diferentes períodos de retorno.
    \item Criar uma nova arquitetura de rede com mais camadas ocultas e outras funções de ativação.
    \item Verificar a capacidade de extrapolação da rede neural artificial.
\end{enumerate}