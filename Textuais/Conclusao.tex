\chapter{Conclusão}

Curta e objetiva. O que devia ser dito, já foi dito. Nessa seção se faz um breve resumo dos melhores resultados do trabalho, reforçando o que já foi apresentado na discussão. Normalmente não passa de uma página, contendo de 2 a 3 parágrafos apenas. A conclusão deve estar rigorosamente alinhada com os objetivos do trabalho. É comum que nessa altura do relatório se saiba exatamente o que o estudo é e, com isso, muitas vezes é conveniente alterar o título do trabalho para que ele expresse o conteúdo do documento com a máxima especificidade possível.
Se oportuno, a critério do Orientador(a) um parágrafo adicional ou até nova seção deve ser incluída com sugestões de continuações ou do estudo de outros aspectos que, ao decorrer do estudo, foram percebidos como relevantes e merecedores de investigação específica.
